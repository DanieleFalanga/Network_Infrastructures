\documentclass[12pt]{article}
\usepackage[top=2cm]{geometry}
\usepackage{amsmath}
\usepackage{amsthm}
\usepackage{amsfonts}
\usepackage{parskip}
\usepackage{graphicx}
\graphicspath{ {images/} }

\title{\textbf{Segment Routing}}
\author{}
\date{}

\begin{document}
\maketitle

Fare troubleshooting in MPLS è veramente complicato, perché se guardo un pacchetto l'informazione 
contenuta nel header trovo una label la quale è un numero, e visto cosi non mi dice nulla riguardo la storia del pacchetto
Per ricotruire la storia del pacchetto dovrei andare a ritroso in ogni nodo. 

Per questo motivo, MPLS è stato sostituito da Segment Routing nello stack dei protocolli. 


\begin{figure*}[h]
    \includegraphics*[scale = 0.5]{F1.png}
    \centering
\end{figure*}

l'obiettivo di Segment Routing è quello di tenere tutti i pro di MPLS ed eliminare i contro. 

\section{Introduction}

Segment routing support source routing paradigm. Si assegna un "identifier"chiamati {\bf{SID}} ,che per ora è un numero,
ad ogni nodo della rete e ad ogni entità della rete.   

Ogni entità della rete può eseguire due tioologie di istruzioni:
\begin{itemize}
    \item Topological istructions
    \item Service based instructions
\end{itemize}

Le istruzioni possono essere:
\begin{itemize}
    \item Local: può essere eseguita solo da un entità in modo locale. Un esempio sarebbe quello di inviare un pacchetto su una specifica interfaccia, solo il router proprietario dell'interfaccia
    può inviare pacchetti su quell'interfaccia
    \item Global: possono essere eseguite da tutti. Un esempio consiste nell'inviare un pacchetto ad un'altra entita
    della rete, può essere fatto da tutti 
\end{itemize}

Segment Routing, a differenza di IP permette di creare un percorso nella rete ben specifico. Permette
di creare una lista ordinata  di destinatari a cui far arrivare il pacchetto. Chiamata \textbf{Segment list}

\begin{figure*}[h]
    \includegraphics*[scale = 0.3]{F2.png}
    \centering
\end{figure*}

Se il percorso verso una destinazione non è specificato manualmente attraverso la lista viene usato the "shortest path" perché l'unico che conosciamo. 
Questo perché Segment Routing è una overlay architecture costruita su un underlay infrastructure che in questo caso è IP. 
Il routing dei pacchetti e gestito dal IP control plane che utilizza come metodologia \textit{the shortest path}. 

Il procedimento è il seguente: 
\begin{enumerate}
    \item Segment Routing crea la \textbf{segment list}
    \item Chiede al control plane quale percorso deve seguire il segmento della segment list
    \item Il segmento viene inviato alla destinazione chiesta dal Segment Routing
    \item il procedimento si ripete con il segmento successivo
\end{enumerate}

Inserire immagine minuto 27

Come possiamo distribuire le informazioni riguardanti gli altri SIDs della rete? In altri termini, come faccio
a sapere dell'esistenza degli altri SIDs della rete?

Questa procedura viene fatta attraverso L'OSPF protocol, quest'ultimo invia un messaggio (LSA, Link state advertisment) agli altri nodi
della rete attraverso i suoi link contentente le seguenti info: 
\begin{itemize}
    \item il SID del sender
    \item I link della rete a cui è collegato
    \item I SIDs che conosce
\end{itemize}  

Ogni nodo della rete, ha un \textbf{topology database} il quale viene riempito con le info in arrivo dagli LSA degli altri SIDs.
Una volta che gli altri nodi ricevono LSA di un nodo, lo inoltrano verso gli altri link, come in figura. 

Inserire figura al minuto 36

Questo procedimento è chiamato \textbf{Flooding}

Il compito di calcolare la segment list è lasciato a un "centralized node" chiamato \textbf{SDN controller}. 
SDN suggerisce di centralizzare l'intelligenza di rete in un componente separato, disassociando il processo di forwarding dei pacchetti (Data Plane) da quello di routing (Control Plane). 
Calcola i percorsi che devono seguire i pacchetti in rete e installa delle SR policy all'interno dei nodi, le quali decidono dove instradare
i pacchetti. 

figura al minuti 41

Attualmente Segment Routing non ha sostituito del tutto MPLS e IPv6, per far si che funzioni tutto, SR ingloba 
le funzioni di entrambi. 

\subsection{SR forwarding}

\begin{figure*}[h]
    \includegraphics*[scale = 0.3]{F5.png}
    \centering
\end{figure*}

Quando arriva in ingresso un pacchetto, i nodi SR hanno una forawrding table che permette
di capire cosa fare con il pacchetto. 
Ogni nodo può fare 3 operazioni
\begin{itemize}
    \item PUSH: Inserisce i segmenti all'inizio della lista dei segmenti 
    \item NEXT: Quando il pacchetto arriva al SID presente nella segment list, il nodo analizza
    il segmento successivo per capire dove inoltrare il pacchetto senza però eliminare il segmento precedente. 
    \item CONTINUE: Questa operazione è svolta dai nodi di "passaggio". Quelli che non sono presenti
    nella segment list e ricevono il pacchetto. Consiste in una operazione di inoltro senza modifica della 
    segment list. 
\end{itemize}

\begin{figure*}[h]
    \includegraphics*[scale = 0.3]{F6.png}
    \centering
    \caption{NEXT operation}
\end{figure*}

\begin{figure*}[h]
    \includegraphics*[scale = 0.3]{F7.png}
    \centering
    \caption{CONTINUE operation}
\end{figure*}


\section{Segment Routing Policy}

Sono istruzioni logiche che devono gli ingress node devono eseguire nei confronti di un pacchetto in arrivo. Sono implementate
negli \textit(edge nodes)

Gli edge nodes hanno un database chiamato \textbf{SR policy database (SRPDB)} che contiene tutte le policy che devono essere applicate
ai rispettivi pacchetti in arrivo. 

Ogni \textit(tupla) del database ha una struttura di questo tipo: 

\[<i,e,color>\]

Dove \(i\) sta per ingress, \(e\) per egress e \(color\) la tipologia di policy da applicare al pacchetto.  

\begin{figure*}[h]
    \includegraphics*[scale = 0.5]{F8.png}
    \centering
\end{figure*}

Quando un pacchetto arriva al nodo \(i\) avviene questo: 
\begin{enumerate}
    \item Il classifier decide quale policy è meglio per il pacchetto
    \item l'SRPDB chiede al Segment List Database (SLDB) quale path tra Quelli
    del colore scelto dal classifier sia meglio per il pacchetto
    \item SRPDB scrive la tupla nel suo database
\end{enumerate}  

\subsection{Bindig SID}

\begin{figure*}[h]
    \includegraphics*[scale = 0.5]{F9.png}
    \centering
\end{figure*}

Ogni tupla del SRPDB può essere assegnata ad uno specifico SID. Quando si riceve in ingresso
una segment list, dove all'interno di essa di essa c'è un SID \textbf{(B)}, il quale è assegnato ad una policy del database, 
Il segmento corrispondente, viene sostituito con la segment list associata alla policy di \textbf{B}. 
Il \textit{bindng SID} è utile perché permette di ridurre la dimensione della segment list. 

\section{SRv6 and Segment Routing extension header}

Come detto prima, Segment Routing è implementato in overlay ad MPLS e IPv6, per favorire
la transizione. 

\begin{figure*}[h]
    \includegraphics*[scale = 0.5]{F10.png}
    \centering
\end{figure*}

Per implementare SR in IPv6, viene aggiunto un header SRH (Segment Routing Header) gestito come 
se fosse un extension header di IPv6. 

\begin{figure*}[h]
    \includegraphics*[scale = 0.5]{F11.png}
    \caption[short]{SRv6 Extension Header}
    \centering
\end{figure*}

Il campo <seg left> è un puntatore allo stack dei segmenti.

L'extension header di SG tratta i SID come indirizzi IPv6 incrementando di gran lunga la dimensione del header.
In MPLS invece, le label hanno di dimensione 20 bit. 

\newpage
Ci sono due modi per aggiungere un extension header ad un 
pacchetto proveniente da un costumer. Dipende da quale protocollo IP viene usato dal costumer 
\begin{itemize}
    \item Se IPv6: l'extension header viene inserito in coda e trattato come se fosse un extension header normale. 
    \item Se IPv4: il pacchetto viene incapsulato  
\end{itemize}


\begin{figure*}[h]
    \includegraphics*[scale = 0.5]{F12.png}
    \centering
\end{figure*}

\subsection{Local Sid Table}

Ogni nodo Srv6 mantiene localmente una tabella locale dei SID, la quale per ognuno di essi, è associata una 
funzione, chiamata \textbf{end behavior} 

\begin{figure*}[h]
    \includegraphics*[scale = 0.5]{F13.png}
    \centering
\end{figure*}

Le funzioni possono essere di due tipologie:
\begin{itemize}
    \item \textbf{End}: È una funzione associata quando il SID di destinazione si trova nella MyLocalSID Table
    \item \textbf{End.X} è una funzione di Layer 3
\end{itemize}


\newpage
\section{Network Programming}

Grazie alle caratteristice spiegante precedentemente è possibile creare una programmazione
di rete, per cui ad ogni SID è possibile associare una funzione corrispondente. 

\begin{figure*}[h]
    \includegraphics*[scale = 0.5]{F14.png}
    \centering
\end{figure*}

In figura è rappresentata la struttura di un indirizzo IPv6 utilizzato in particolare nel protocollo, 
SRv6. I primi 64 bits dell'inidirizzo vengono uasti come localizzatore del SID. I campi 
\textit{function} e \textit{argument} sono campi usati dalla MyLocalSID Table del \textit{locator} per 
inidirizzare il pacchetto verso il nodo che performerà la funzione corrispondente. 



\end{document}